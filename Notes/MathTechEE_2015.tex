%%% DOCUMENTCLASS 
%%%-------------------------------------------------------------------------------

\documentclass[
letterpaper, %a4paper, % Stock and paper size.
12pt, % Type size.
% article,
% oneside, 
onecolumn, % Only one column of text on a page.
% openright, % Each chapter will start on a recto page.
% openleft, % Each chapter will start on a verso page.
openany, % A chapter may start on either a recto or verso page.
]{memoir}

%%% PACKAGES 
%%%------------------------------------------------------------------------------

\usepackage[utf8]{inputenc} % If utf8 encoding
% \usepackage[lantin1]{inputenc} % If not utf8 encoding, then this is probably the way to go
\usepackage[T1]{fontenc}    %
\usepackage[english]{babel} % English please
\usepackage[final]{microtype} % Less badboxes

% \usepackage{kpfonts} %Font

\usepackage{amsmath,amssymb,mathtools} % Math
\usepackage{commath} % Improved mathematical notation

\usepackage{tikz} % Figures
\usepackage{graphicx} % Include figures

%%% PAGE LAYOUT 
%%%------------------------------------------------------------------------------

\setlrmarginsandblock{0.15\paperwidth}{*}{2} % Left and right margin {<spine>}{<edge>}{<ratio>}
\setulmarginsandblock{0.2\paperwidth}{*}{1}  % Upper and lower margin
\checkandfixthelayout

%%% SECTIONAL DIVISIONS
%%%------------------------------------------------------------------------------

\maxsecnumdepth{subsection} % Subsections (and higher) are numbered
\setsecnumdepth{subsection}

\makeatletter %
\makechapterstyle{standard}{
  \setlength{\beforechapskip}{0\baselineskip}
  \setlength{\midchapskip}{1\baselineskip}
  \setlength{\afterchapskip}{8\baselineskip}
  \renewcommand{\chapterheadstart}{\vspace*{\beforechapskip}}
  \renewcommand{\chapnamefont}{\centering\normalfont\Large}
  \renewcommand{\printchaptername}{\chapnamefont \@chapapp}
  \renewcommand{\chapternamenum}{\space}
  \renewcommand{\chapnumfont}{\normalfont\Large}
  \renewcommand{\printchapternum}{\chapnumfont \thechapter}
  \renewcommand{\afterchapternum}{\par\nobreak\vskip \midchapskip}
  \renewcommand{\printchapternonum}{\vspace*{\midchapskip}\vspace*{5mm}}
  \renewcommand{\chaptitlefont}{\centering\bfseries\LARGE}
  \renewcommand{\printchaptertitle}[1]{\chaptitlefont ##1}
  \renewcommand{\afterchaptertitle}{\par\nobreak\vskip \afterchapskip}
}
\makeatother

\chapterstyle{standard}

\setsecheadstyle{\normalfont\large\bfseries}
\setsubsecheadstyle{\normalfont\normalsize\bfseries}
\setparaheadstyle{\normalfont\normalsize\bfseries}
\setparaindent{0pt}\setafterparaskip{0pt}

%%% FLOATS AND CAPTIONS
%%%------------------------------------------------------------------------------

\makeatletter                  % You do not need to write [htpb] all the time
\renewcommand\fps@figure{htbp} %
\renewcommand\fps@table{htbp}  %
\makeatother                   %

\captiondelim{\space } % A space between caption name and text
\captionnamefont{\small\bfseries} % Font of the caption name
\captiontitlefont{\small\normalfont} % Font of the caption text

\changecaptionwidth          % Change the width of the caption
\captionwidth{1\textwidth} %

%%% ABSTRACT
%%%------------------------------------------------------------------------------

\renewcommand{\abstractnamefont}{\normalfont\small\bfseries} % Font of abstract title
\setlength{\absleftindent}{0.1\textwidth} % Width of abstract
\setlength{\absrightindent}{\absleftindent}

%%% HEADER AND FOOTER 
%%%------------------------------------------------------------------------------

\makepagestyle{standard} % Make standard pagestyle

\makeatletter                 % Define standard pagestyle
\makeevenfoot{standard}{}{}{} %
\makeoddfoot{standard}{}{}{}  %
\makeevenhead{standard}{\bfseries\thepage\normalfont\qquad\small\leftmark}{}{}
\makeoddhead{standard}{}{}{\small\rightmark\qquad\bfseries\thepage}
% \makeheadrule{standard}{\textwidth}{\normalrulethickness}
\makeatother                  %

\makeatletter
\makepsmarks{standard}{
\createmark{chapter}{both}{shownumber}{\@chapapp\ }{ \quad }
\createmark{section}{right}{shownumber}{}{ \quad }
\createplainmark{toc}{both}{\contentsname}
\createplainmark{lof}{both}{\listfigurename}
\createplainmark{lot}{both}{\listtablename}
\createplainmark{bib}{both}{\bibname}
\createplainmark{index}{both}{\indexname}
\createplainmark{glossary}{both}{\glossaryname}
}
\makeatother                               %

\makepagestyle{chap} % Make new chapter pagestyle

\makeatletter
\makeevenfoot{chap}{}{\small\bfseries\thepage}{} % Define new chapter pagestyle
\makeoddfoot{chap}{}{\small\bfseries\thepage}{}  %
\makeevenhead{chap}{}{}{}   %
\makeoddhead{chap}{}{}{}    %
% \makeheadrule{chap}{\textwidth}{\normalrulethickness}
\makeatother

\nouppercaseheads
\pagestyle{standard}               % Choosing pagestyle and chapter pagestyle
\aliaspagestyle{chapter}{chap} %

%%% NEW COMMANDS
%%%------------------------------------------------------------------------------

\newcommand{\p}{\partial} %Partial
% Or what ever you want

%%% TABLE OF CONTENTS
%%%------------------------------------------------------------------------------

\maxtocdepth{subsection} % Only parts, chapters and sections in the table of contents
\settocdepth{subsection}

\AtEndDocument{\addtocontents{toc}{\par}} % Add a \par to the end of the TOC

%%% INTERNAL HYPERLINKS
%%%-------------------------------------------------------------------------------

\usepackage{hyperref}   % Internal hyperlinks
\hypersetup{
pdfborder={0 0 0},      % No borders around internal hyperlinks
pdfauthor={I am the Author} % author
}
\usepackage{memhfixc}   %

%%% THE DOCUMENT
%%% Where all the important stuff is included!
%%%-------------------------------------------------------------------------------

\author{Simon Aeschbacher}
\title{Mathematical Techniques in\\Evolution and Ecology}

\usepackage{lipsum} % Just to put in some text

\begin{document}

\frontmatter

\maketitle

%\begin{abstract}
%\lipsum[1-2]
%\end{abstract}
\clearpage

\tableofcontents*
\clearpage

\chapter{Introduction}
To come.

\mainmatter

\chapter{How to construct a model}

\section{Problems}

\begin{enumerate}
	\item[P.1.1] In the squirrel example, assume that death through cyclists occurs before immigration, and immigration is followed by birth. Derive the discrete-time recursion equation and compare it to the one we obtained above, where we assumed that death is followed by birth and immigration. Start from the life-cycle diagram. Interpret the difference.
	\item[P.1.2] [Problem 2.5 in OD2007] Consider a model of disease transmission with the following equations:
		\begin{subequations}
			\label{eq:prob1-1}
			\begin{align}
				\frac{dS}{dt} & = \theta - d D - \beta S I + \gamma I,\\
				\frac{dI}{dt} & = \beta S I - (d + \nu + \gamma) I.
			\end{align}
		\end{subequations}
		The variables $S$ and $I$ denote the number of susceptible and infected individuals. (a) Draw and label a flow diagram for these two variables. (b) Suggest a plausible biological interpretation of the parameters $\gamma$ and $\nu$.
	\item[P.1.3] [Problem 2.6 in OD2007] In the flu model, suppose that after contracting the flu, people are initially resistant to reinfection, but this immunity eventually wanes. Alter the flow diagram for the flu model that we derived to include a ``recovered and immune'' class with these properties. (b) Suppose that immune individuals have a constant per capita rate of losing immunity. What are the continuous-time equations for this modified model?
\end{enumerate}

\chapter{Some classic models in evolution and ecology}

\section{Problems}

\begin{enumerate}
	\item[P.2.1] We constructed the differential equation for the continuous-time version of the exponential growth model from the flow diagram. (a) Instead, derive the equation from the difference equation of the discrete-time version. Hint: Introduce a small amount of time $\Delta t$ and take the appropriate limit. (b) Comparing $r_d$ and $r_c$, explain the difference between the discrete- and continuous-time versions.
	
	\item[P.2.2] Derive the differential equation for the logistic growth model in continuous time. Start by assuming that the per capita growth rate $r(n)$ declines linearly with population size $n(t)$. Assume that $r(n)$ has a maximum equal to the intrinsic growth rate $r_c$ when there are no competitors (i.e.\ $r(0) = r_c$), and that it reaches zero when the population is at the carrying capacity (i.e.\ $r(K) = 0$). Substitute $r(n)$ for $r_c$ in the differential equation of the exponential growth model to obtain the result of interest.
	
	\item[P.2.3] As an alternative way of deriving the continuous-time version of the haploid selection model, start from the continous-time version of the exponential growth model. Specifically, let $n_A$ and $n_a$ be the number of individuals carrying alleles $A$ and $a$, respectively, and let the growth rate depend on the allelic state, i.e.\ $r_A = b_A - d_A$ for carriers of $A$, and $r_a = b_a - d_a$ for carriers of $a$. From the exponential growth model, we have
	\begin{subequations}
		\label{eq:haplselconttime}
		\begin{align}
			\frac{d n_A}{dt} &= r_A n_A(t),\\
			\frac{d n_a}{dt} &= r_a n_a(t).
		\end{align}
	\end{subequations}
	However, we want to know how the allele frequencies change! As these sum to 1, it is sufficient to follow only the frequency of allele $A$, $p(t)=n_A(t) / (n_A(t) + n_a(t))$. Start by writing
	\begin{equation}
		\label{eq:hapmodcont}
		\frac{dp}{dt} = \frac{d\left[\frac{n_A(t)}{n_A(t) + n_a(t)}\right]}{dt}.
	\end{equation}
	Then use the quotient rule to evaluate the derivative, and use the equations for $d n_A / d t$ and $d n_a / d t$ to express $dp/dt$ in terms of $r_A$, $r_a$, and $p$.
\end{enumerate}

\chapter{Primer 1: Functions and approximations}

\section{Problems}

\begin{enumerate}
	\item[P.3.1] Determine the functions $R(n)$ for the reproductive factor in the logistic growth model, such that the shape of the functions is consistent with the first figure above. In each case, choose the parameters such that the intercept is $R(0)=1+r$, and the value when $N=K$ is $R(K)=1$. Specifically, derive
	\begin{enumerate}
		\item[(a)] A function that declines exponentially to zero.
		\item[(b)] A quadratic function with a maximum at $n=0$.
		\item[(c)] A reverse-S-shaped function that declines from $1+r$ to zero. Keep $a$ arbitrary, and use Rule P1.1 above to increase the intercept to $1+r$. Hint: Start by stretching the generic formula for the S-shaped function along the $y$ axis.
	\end{enumerate}
	
	\item[P.3.2] Use Recipe P1.1 to confirm the following linear approximations:
	\begin{enumerate}
		\item[(a)] $e^r \approx 1 + r$, assuming $r$ is small,
		\item[(b)] $1/(1+s) \approx 1 - s$, assuming $s$ is small,
		\item[(c)] $\ln(t) \approx t - 1$, assuming $t$ is near 1, and
		\item[(d)] $1/x \approx 1/a - 1/(a^2) (x - a)$, assuming $x$ is near $a$.
	\end{enumerate}
\end{enumerate}

\chapter{Solutions to problems}

\section{Chapter 1}

\begin{enumerate}
	\item[P.1.1] The life-cycle diagram is as given in Fig.\ \ref{fig:lcdsquirrelsalt}. The recursion equation is
		\begin{equation}
			n(t+1) = (1+b) \left[(1-d)n(t) + m\right],
		\end{equation}
		as compared to
		\begin{equation}
			n(t+1) = (1+b)(1-d) n(t) + m
		\end{equation}
		for the case of immigration after death and birth. The first equation differs from the second one by an amount of $bm$, which is the fraction of squirrels born by immigrating (female) squirrels.
		\begin{figure}[!bth]
			\begin{center}
			\begin{tikzpicture}[scale = 1.5]
				% circle and arrow heads
				\draw[->, thick] (-10mm, 0mm) arc (180:90:10mm);
				\draw[->, thick] (0mm, 10mm) arc (90:0:10mm);
				\draw[->, thick] (10mm, 0mm) arc (0:-90:10mm);
				\draw[->, thick] (0mm, -10mm) arc (-90:-180:10mm);
				
				% events
				\draw (0, 10mm) node[anchor=south] {census};
				\draw (10mm, 0mm) node[anchor=west] {death};
				\draw (0mm, -10mm) node[anchor=north] {immigration};
				\draw (-10mm, 0mm) node[anchor=east] {birth};
				
				% variables
				\draw (0, 10mm) node[anchor=north] {$n$};
				\draw (10mm, 0mm) node[anchor=east] {$n'$};
				\draw (0mm, -10mm) node[anchor=south] {$n''$};
				\draw (-10mm, 0mm) node[anchor=west] {$n'''$};
				
			\end{tikzpicture}
			\end{center}
			\caption{Life-cycle diagram for a squirrel model with alternative order of events.}
			\label{fig:lcdsquirrelsalt}
		\end{figure}
	\item[P.1.2] The flow diagram is shown in Fig.\ \ref{fig:fldismodel}. The parameter $\gamma$ is the rate at which infected individuals recover and become susceptible again. The parameter $\nu$ denotes an additional death rate experienced by infected individuals as compared to the death rate $d$ of susceptible ones.
		\begin{figure}[!bth]
			\begin{center}
			\begin{tikzpicture}[scale = 1.1]
				% circles
				\draw[thick] (-30mm, 0mm) circle (10mm);
				\draw[thick] (30mm, 0mm) circle (10mm);
				
				% arrows
				\draw[->, thick] (-20mm, 2mm) -- (20mm, 2mm);
				\draw[->, thick] (20mm, -2mm) -- (-20mm, -2mm);
				\draw[<-, thick] (-37.07107mm, 7.07107mm) -- (-41.3137mm, 11.3137mm);
				\draw[->, thick] (-37.07107mm, -7.07107mm) -- (-41.3137mm, -11.3137mm);
				\draw[->, thick] (37.07107mm, -7.07107mm) -- (41.3137mm,  -11.3137mm);
				\draw[thick, dashed] (28mm, 10mm) arc (0:270:8mm);
				
				% variables
				\draw (-30mm, 0mm) node {$S(t)$};
				\draw (30mm, 0mm) node {$I(t)$};
				
				% flows
				\draw (-42mm, 8mm) node {$\theta$};
				\draw (-44mm, -7mm) node {$d S(t)$};
				\draw (48mm, -7mm) node {$(d + \nu) I(t)$};
				\draw (-7.5mm, -4.5mm) node {$\gamma I(t)$};
				\draw (5mm, 4.5mm) node {$\beta S(t) I(t)$};
				
			\end{tikzpicture}
			\end{center}
			\caption{Flow diagram for the continuous-time model of Problem P.1.2.}
			\label{fig:fldismodel}
		\end{figure}
	\item[P.1.3] The flow diagram is as given in Fig.\ \ref{fig:lsdirsmodel}, and the continuous-time differential equations are
		\begin{subequations}
			\label{eq:lsdirsmodel}
			\begin{align}
				\frac{ds}{dt} &= \sigma r(t) - a c s(t) n(t)\\
				\frac{dn}{dt} &= a c s(t) n(t) - \rho n(t)\\
				\frac{dr}{dt} &= \rho n(t) - \sigma r(t)
			\end{align}
		\end{subequations}
		\begin{figure}[!bth]
			\begin{center}
			\begin{tikzpicture}[scale = 1.1]
				% circles
				\draw[thick] (-30mm, 0mm) circle (10mm);
				\draw[thick] (30mm, 0mm) circle (10mm);
				\draw[thick] (0mm, -30mm) circle (10mm);
				
				% arrows
				\draw[->, thick] (-20mm, 0mm) -- (20mm, 0mm);
				\draw[<-, thick] (-22.92803mm, -7.07107mm) -- (-7.07107mm, -22.92003mm);
				\draw[->, thick] (22.92803mm, -7.07107mm) -- (7.07107mm, -22.92003mm);
				\draw[thick, dashed] (28mm, 10mm) arc (10:270:8.5mm);
				
				% variables
				\draw (-30mm, 0mm) node {$s(t)$};
				\draw (30mm, 0mm) node {$n(t)$};
				\draw (0mm, -30mm) node {$r(t)$};
				
				% flows
				\draw (-22mm, -16mm) node {$\sigma r(t)$};
				\draw (24mm, -16mm) node {$\rho n(t)$};
				\draw (2.5mm, 3mm) node {$a c s(t) n(t)$};
				
			\end{tikzpicture}
			\end{center}
			\caption{Flow diagram for the continuous-time version of the flu model of Problem P.1.3, with an additional class of resistant individuals.}
			\label{fig:lsdirsmodel}
		\end{figure}
\end{enumerate}


\section{Chapter 2}

\begin{enumerate}
	\item[P.2.1] We start from the discrete-time difference equation for the exponential growth model,
		\begin{equation}
			\Delta n = n(t+1) - n(t) = R n(t) - n(t) = (R - 1) n(t)
		\end{equation}
		We recall that the reproductive factor was given by $R = (1+b)(1-d)$, where $b$ and $d$ are the number of offspring per adult female and the proportion of individuals dying per generation, respectively. Moreover, we recall that the growth rate was defined as $r_d = R - 1 = b - d - b d$. Therefore, the difference equation can be written as
		\begin{equation}
			\Delta n = (R - 1) n(t) = r_d n(t) = (b - d - b d) n(t).
		\end{equation}
		To derive the differential equation, consider a small amount of time $\Delta t$, such that the number of offspring per adult female and the proportion of individuals dying during this time are now $b \Delta t$ and $d \Delta t$, respectively. Hence, $n(t+\Delta t) - n(t) = \left[b \Delta t - d \Delta t - b d (\Delta t)^2  \right] n(t)$. Applying the definition of the differential, we find
		\begin{equation}
		\begin{split}
			\frac{dn}{dt} & = \lim_{\Delta t \rightarrow 0} \frac{n(t+\Delta t) - n(t)}{\Delta t}\\
			& =  \lim_{\Delta t \rightarrow 0} \frac{\left[b \Delta t - d \Delta t - b d (\Delta t)^2  \right] n(t)}{\Delta t} \\
			& =  \lim_{\Delta t \rightarrow 0} (b - d - b d \Delta t) n(t)\\
			& = (b - d) n(t).
		\end{split}
		\end{equation}
		This implies that the continuous-time growth rate is $r_c = b - d$. Comparing to $r_d = b - d - bd$, we see that the interaction term $b d$ is lost when we switch from discrete to continuous time. This is because, in the limit of a time step becoming small, no more than one type of event occurs. In continuous time, individuals that are born in a small amount of time will not suffer death in that same amount of time.
		
		\item[P.2.2] We start by recalling the continuous-time differential equation for the exponential growth model,
		\begin{equation}
			\label{eq:expgrowthcont}
			\frac{dn}{dt} = r_c n(t).
		\end{equation}
		To account for the effect of intra-specific competition for limited resources, we want $r_c$ to depend on population size as described in the problem. For this, the intercept needs to be equal to $r_c$ and the slope equal to $-r_c/K$. Hence,
		\begin{equation}
			r_c(n) = r_c + \left(-\frac{r_c}{K}\right) n(t) = r_c \left( 1 - \frac{n(t)}{K}\right).
		\end{equation}
		Substituting $r_c(n)$ for $r_c$ in Eq.\ \eqref{eq:expgrowthcont}, we obtain the solution
		\begin{equation}
			\frac{dn}{dt} = r_c \left( 1 - \frac{n(t)}{K}\right) n(t).
		\end{equation}
		
		\item[P2.3] Applying the quotient rule of differentiation, we find
		\begin{equation}
		\begin{split}
			\frac{dp}{dt} & = \frac{\frac{d n_A}{dt} \left(n_A + n_a \right) - n_A \left(\frac{d n_A}{dt} + \frac{d n_a}{dt} \right)}{\left(n_A + n_a \right)^2}\\
			& = \frac{n_a \frac{d n_A}{dt} - n_A \frac{d n_a}{dt}}{\left(n_A + n_a \right)^2}.
		\end{split}
		\end{equation}
		Substitution of $d n_A / dt = r_A n_A(t)$ and $d n_a / dt = r_a n_a(t)$ yields
		\begin{equation}
		\begin{split}
			\frac{dp}{dt} & = \frac{n_a (r_A n_A) - n_A (r_a n_a)}{\left(n_A + n_a \right)^2} \\
			& = (r_A - r_a) \frac{n_A n_a}{\left(n_A + n_a \right)^2} = (r_A - r_a) p(t) q(t),
		\end{split}
		\end{equation}
		using $q(t) = 1 - p(t) = n_a(t) / (n_A(t) + n_a(t))$. Note that in the derivation we have omitted the dependence of $n_A$ and $n_a$ on $t$ in our notation for simplicity. Comparing the result to the differential equation derived earlier, $dp / dt = s_c p(t) q(t)$, we see that the continuous-time selection coefficient $s_c$ can be interpreted as the difference between the intrinsic growth rates of the two haploid genotypes that compete.
\end{enumerate}

\section{Primer 1}

\begin{enumerate}
	
	\item[P.3.1]
		\begin{itemize}
			\item[(a)] Start by writing $R(n)$ as a generic exponential function, $R(n) = c e^{-b n}$, where $c$ is the intercept and $b$ is the rate of decline. Note that when writing the $-$ in front of $b$, for there to be a decline, $b$ has to be positive. We are told that the intercept is $c = R(0) = 1 + r$, and that $R(K) = (1+r) e^{-b K} = 1$. Hence, the slope can be found by solving for $b$:
			\begin{equation}
				\begin{split}
					(1+r) e^{-b K}& = 1\\
					&\Rightarrow e^{-b K} = \frac{1}{1+r}\\
					&\Rightarrow -b K = - \ln(1+r)\\
					&\Rightarrow b = \frac{\ln(1+r)}{K}.
				\end{split}
			\end{equation}
			Putting things together, we obtain
			\begin{equation}
				\begin{split}
					 R(n) &= (1+r) e^{-\ln(1+r)n/K}\\
					& = (1+r)(1+r)^{-n/K}\\
					& = (1+r)^{1-n/K}.
				\end{split}
			\end{equation}
			
			\item[(b)] Start by writing $R(n)$ as a generic quadratic function, $R(n) = a n^2 + bn + c$, where $c$ is the intercept and $b$ and $a$ are the coefficients of the linear and quadratic term, respectively. We are told that the intercept is $c = R(0) = 1 + r$, and that $R(K) = a K^2 + bK + c = 1$. Moreover, we are told that $R(0)$ is a maximum, which implies that $b = 0$. To find $a$, we need to solve $R(K) = a K^2 + bK + c = a K^2 + (1+r) = 1$ for $a$:
			\begin{equation}
				\begin{split}
					 a K^2 + 1 + r &= 1\\
					&\Rightarrow  a K^2 = -r \\
					&\Rightarrow a = -\frac{r}{K^2}.
				\end{split}
			\end{equation}
			Putting things together, we obtain
			\begin{equation}
				\begin{split}
					R(n) &= 1+r - \frac{r}{K^2}n^2.
				\end{split}
			\end{equation}
			
			\item[(c)] Start by writing $R(n)$ as a generic sigmoidal function,
				\begin{equation}
					R(n) = \frac{c \, e^{an}}{c \, e^{a n} + (1+c)},
				\end{equation}
			where $a$ should be kept general, and $c$ tells us how far up the way to 1 the function starts when $n=0$. We are told that the intercept is $R(0) = 1 + r$. Therefore, we start by stretching the function along the $y$ axis by a factor of $(1+r)/c$, which yields
			\begin{equation}
				\begin{split}
					R(n) &= (1+r)\ \frac{e^{an}}{c\, e^{an} + 1 - c}.
				\end{split}
			\end{equation}
			We also know that $R(K) = 1$, which allows us to identify $c$ as follows:
			\begin{equation}
				\begin{split}
					R(K) &= \frac{(1+r)\, e^{aK}}{c\, e^{aK} + 1 - c} = 1\\
					& \Rightarrow c \left( e^{aK} + 1 \right) + 1 = (1+r)\, e^{aK}\\
					& \Rightarrow c = \frac{(1+r)\, e^{aK} - 1}{e^{aK} - 1} \\
					& \Rightarrow c = \frac{e^{aK} - 1 + r\ e^{a K}}{e^{aK} - 1} \\
					& \Rightarrow c = 1 - \frac{r\, e^{aK}}{1 - e^{aK}}.
				\end{split}
			\end{equation}
			Putting things together, we find
			\begin{equation}
				R(n) = (1+r) \frac{e^{an}}{\left( 1 - \frac{r\, e^{aK}}{1 - e^{aK}} \right)\, e^{an} + \frac{r\, e^{aK}}{1 - e^{aK}}}.
			\end{equation}
			
		\end{itemize}
		
	\item[P.3.2]
		\begin{itemize}
			
			\item[(a)] We expand $f(r) = e^r$ as a Taylor series around $r = 0$ by setting
				\begin{equation}
				\begin{split}
					f(r) &= f(0) + \left . \frac{df}{dr}\right |_{r=0} (r - 0) + \mathcal{O}(r^2) \\
					&= 1 + e^0 r + \mathcal{O}(r^2)\\
					& \approx 1 + r.
				\end{split}
				\end{equation}
				
			\item[(b)] We expand $f(s) = 1/(1+s)$ as a Taylor series around $s = 0$ by setting
				\begin{equation}
				\begin{split}
					f(r) &= f(0) + \left . \frac{df}{ds}\right |_{s=0} (s- 0) + \mathcal{O}(s^2) \\
					&= 1 + (-1)(s-0) + \mathcal{O}(s^2)\\
					& \approx 1 - s.
				\end{split}
				\end{equation}
				
			\item[(c)] We expand $f(t) = \ln(t)$ as a Taylor series around $t = 1$ by setting
				\begin{equation}
				\begin{split}
					f(t) &= f(1) + \left . \frac{df}{dt}\right |_{t=1} (t - 1) + \mathcal{O}(t^2) \\
					&= \ln(1) + 1(t-1) + \mathcal{O}(t^2)\\
					& \approx t - 1.
				\end{split}
				\end{equation}
				
			\item[(d)] We expand $f(x) = 1/x$ as a Taylor series around $x = a$ by setting
				\begin{equation}
				\begin{split}
					f(x) &= f(a) + \left . \frac{df}{dx}\right |_{x=a} (x - a) + \mathcal{O}(x^2) \\
					&= \frac{1}{a} + (-1) \frac{1}{a^2} (x-a) + \mathcal{O}(x^2)\\
					& \approx \frac{1}{a} - \frac{1}{a^2} (x-a).
				\end{split}
				\end{equation}
			
			
		\end{itemize}
	
\end{enumerate}

\chapter{Some rules from calculus}

\section{Derivatives}
The following rules describe some of the more important derivatives.

\begin{description}
	\item[Derivative of a constant]
		\begin{equation}
			\frac{da}{dx} = 0
		\end{equation}
	
	\item[Factoring out a constant]
		\begin{equation}
			\frac{d\left(a f(x)\right)}{dx} = a \frac{d f(x)}{dx}
		\end{equation}
		
	\item[Linearity property]
		\begin{equation}
			\frac{d\left( f(x) + g(x) \right)}{dx} = \frac{df(x)}{dx} + \frac{dg(x)}{dx}
		\end{equation}
		
	\item[Linear functions]
		\begin{equation}
			\frac{d(ax)}{dx} = a
		\end{equation}
		
	\item[Polynomial functions]
		\begin{equation}
			\frac{dx^a}{dx} = a x^{a-1}
		\end{equation}
		
	\item[Exponential functions]
		\begin{equation}
			\frac{d\left( e^{f(x)} \right)}{dx} = e^{f(x)} \frac{df(x)}{dx}
		\end{equation}
		
	\item[Power functions I]
		\begin{equation}
			\frac{d\left( a^{f(x)} \right)}{dx} = a^{f(x)} \frac{d f(x)}{dx} \ln(a)
		\end{equation}
		
	\item[Power functions II]
		\begin{equation}
			\frac{d\left( f(x)^a \right)}{dx} = a f(x)^{a-1} \frac{df}{dx}
		\end{equation}
		
	\item[Natural log functions]
		\begin{equation}
			\frac{d\left( \ln\left( f(x) \right) \right)}{dx} = \frac{1}{f(x)} \frac{df(x)}{dx}
		\end{equation}
	
	\item[Log functions in base $b$]
		\begin{equation}
			\frac{d\left( \log_{b}\left( f(x) \right) \right)}{dx} = \frac{1}{\ln(b)\, f(x)} \frac{df(x)}{dx}
		\end{equation}
	
	\item[Sine functions]
		\begin{equation}
			\frac{d \left( \sin \left( f(x) \right) \right)}{dx} = \cos \left( f(x) \right) \frac{d f(x)}{dx}
		\end{equation}
		
	\item[Cosine functions]
		\begin{equation}
			\frac{d \left( \cos \left( f(x) \right) \right)}{dx} = -\sin \left( f(x) \right) \frac{d f(x)}{dx}
		\end{equation}
		
	\item[Product rule]
		\begin{equation}
			\frac{d\left( f(x)\, g(x) \right)}{dx} = \frac{df(x)}{dx} g(x) + f(x) \frac{dg(x)}{dx}
		\end{equation}
		
	\item[Quotient rule]
		\begin{equation}
			\frac{d \left( \frac{f(x)}{g(x)} \right)}{dx} = \frac{\frac{df(x)}{dx} g(x) - f(x) \frac{dg(x)}{dx}}{g(x)^2}
		\end{equation}
		
	\item[Chain rule with 1 variable]
		\begin{equation}
			\frac{d \left( f\left( x(t) \right) \right)}{dt} = \frac{df(x)}{dx} \frac{dx(t)}{dt}
		\end{equation}
		
	\item[Chain rule with 2 variables]
		\begin{equation}
			\frac{d \left( f\left( x(t), y(t) \right) \right)}{dt} = \frac{\partial f(x)}{\partial x} \frac{dx(t)}{dt} + \frac{\partial f(y)}{\partial y} \frac{dy}{dt}
		\end{equation}
		
\end{description}

\section{Integrals}

To come.

\backmatter

%%% BIBLIOGRAPHY
%%% -------------------------------------------------------------

% \bibliographystyle{utphysics}
% \bibliography{ref}

\end{document}